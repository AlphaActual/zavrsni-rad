\section{Analiza rezultata}

U ovom poglavlju slijedi detaljna analiza rezultata testiranja radnih značajki tri programska okvira (Next.js, Nuxt.js i SvelteKit) uz četiri strategije iscrtavanja (SSG, SSR, CSR i ISR) na 3 različita tipa stranica. Analiza se temelji na ocjenama  Web Vitals metrika dobivenih mjerenjem Lighthouse CLI alatom: First Contentful Paint (FCP), Largest Contentful Paint (LCP), Speed Index (SI), Time to Interactive (TTI), Total Blocking Time (TBT) i Cumulative Layout Shift (CLS).

\subsection{Analiza ocjena radnih značajki programskih okvira}

Evaluacija ukupnih rezultata radnih karakteristika programskih okvira otkriva jasnu hijerarhiju pri svim vrstama stranica.

\textbf{Next.js} se dosljedno pokazao kao najbolji programski okvir s ocjenama između 92,3\% i 96,0\% ovisno o tipu stranice. Okvir demonstrira najbolje rezultate na statičnom sadržaju (stranica O nama - 96,0\%), što sugerira optimizaciju za statične stranice. Blagi pad u ocjenama kod dinamičkog sadržaja (Blog - 95,1\%, Blog post - 92,3\%) sugerira određeno opterećenje povezano s obradom dinamičkih podataka, no Next.js i dalje zadržava prednost od oko 6-8\% u odnosu na konkurenciju.

\textbf{Nuxt.js} zauzima drugu poziciju s ocjenama između 88,1\% i 89,4\%. Iz rezultata vidljivo je da Nuxt.js pokazuje relativno stabilne performanse kroz različite tipove stranica, s najmanjom varijacijom između statičnog (89,4\%) i dinamičkog sadržaja (88,1\% - 88,5\%). Ova konzistentnost ukazuje na dobro uravnoteženu arhitekturu koja se jednako dobro nosi s različitim tipovima sadržaja.

\textbf{SvelteKit} na testovima bilježi neznatno niži rezultat od Nuxt.js-a (87,4\% - 87,8\%). Baš kao i Nuxt.js, SvelteKit održava konstantne performanse neovisno o vrsti stranice. Manje ocjene mogu se objasniti time da je SvelteKit mlađi programski okvir koji je još uvijek u razvoju i procesu optimizacije.

Kad su u pitanju standardne devijacije, Next.js demonstrira najbolje ocjene dosljednosti (±7,0\% do ±14,8\%), dok su preostali okviri podložniji većim varijacijama (±13,2\% do ±14,1\%). To ukazuje na zrelost Next.js-a kao stabilnijeg i pouzdajnijeg programskog okvira.

\subsection{Analiza ocjena strategija iscrtavanja}

Analiza strategija iscrtavanja otkriva zanimljive obrasce koji se razlikuju ovisno o tipu stranice i prirodi sadržaja.

\textbf{Incremental Static Regeneration (ISR)} ostvaruje najbolje rezultate na statičnoj stranici O nama (91,9\%), što i nije iznenađujuće s obzirom na prednosti statičke generacije uz mogućnost povremenog osvježavanja sadržaja. Na dinamičnim stranicama ISR se drži vrlo stabilno, s rezultatima između 90,6\% i 90,7\%, što potvrđuje njegovu fleksibilnost.

\textbf{Server-Side Rendering (SSR)} pokazao je odlične rezultate, osobito kod prikaza dinamičkog sadržaja na stranicama Blog (91,3\%) i Blog post (90,8\%). Zanimljivo, upravo na statičnoj stranici O nama SSR je ostvario gotovo najbolji rezultat (91,7\%), što govori da mu tip sadržaja ne predstavlja značajnu prepreku.

\textbf{Static Site Generation (SSG)} zadržava dobre ocjene na svim testiranim stranicama, pri čemu se najbolje pokazao na Blogu i O nama (oko 90,6\%–90,7\%). Ta ujednačenost ukazuje na glavnu prednost SSG-a: sadržaj je u potpunosti pripremljen tijekom izgradnje stranice, što rezultira predvidljivim performansama, bez obzira na složenost same stranice.

\textbf{Client-Side Rendering (CSR)} s druge strane, pokazuje znatno veće varijacije. Na O nama stranici postiže solidnih 89,7\%, dok na složenijim dinamičkim stranicama, poput pojedinačnih blog postova, padne i na 85,8\%. Štoviše, zabilježena je i najniža pojedinačna ocjena u cijelom testiranju — samo 40 bodova na blog post stranici (Next.js). Iz grafova
na slici \ref{fig:testiranje-blog-post-performanse-po-metrici} i slici \ref{fig:testiranje-blog-post-postotak} vidljivo je da Next.js programski okvir postiže vrlo loš rezultat metrike CLS specifično na blog post stranici. Ova loša ocjena utječe na cjelokupni rezultat učinkovitosti ove strategije. S obzirom da se na stranici Blog koja je također dinamičnog sadržaja ne pojavljuje ova anomalija, zaključujem da arhitektura koda ove stranice nije optimalno strukturirana za ovu kombinaciju programskog okvira i strategije iscrtavanja.

Kad se promatraju standardne devijacije, SSR i ISR imaju najmanje varijacije (oko ±11,5\% do ±12,9\%), dok CSR pokazuje najveće (±13,0\% do ±17,3\%). To potvrđuje da su server-side strategije u prosjeku predvidljivije i stabilnije.

\subsection{Analiza rezultata pojedinih metrika - Web Vitals}

Detaljnom analizom pojedinačnih Web Vitals metrika možemo identificirati specifične snage i slabosti različitih kombinacija okvira i strategija.

\textbf{Cumulative Layout Shift (CLS)} postiže visoke ukupne rezultate s ocjenama od 95,0\% do 100,0\%, a blog stranica postiže savršenu ocjenu (100,0\%). Drastično veliki pad ocjene ove metrike bilježi Next.js s CSR strategijom na blog post stranici, gdje je zabilježena vrijednost od 0,3 (40 bodova). Ovaj pad ukazuje na značajan problem s pomakom rasporeda elemenata na stranici, što može negativno utjecati na korisničko iskustvo.

\textbf{Total Blocking Time (TBT)} - odlični rezultati (99,9\% - 100,0\%) pri svim kobinacijama demonstriraju dobru optimizaciju programskih okvira. Pogledom na grafove sa vrijednostima ove metrike  (slika \ref{fig:testiranje-o-nama-vrijednosti}, slika \ref{fig:testiranje-blog-vrijednosti} i slika \ref{fig:testiranje-blog-post-vrijednosti}) uočavamo da Next.js programski okvir ima povišeno vrijeme blokiranja u odnosu na druge programske okvire, no pogled na grafove sa ukupnom ocjenom (slika \ref{fig:testiranje-o-nama-postotak}, slika \ref{fig:testiranje-blog-postotak} i slika \ref{fig:testiranje-blog-post-postotak}) otkriva da se ova odstupanja i dalje smatraju odličnim rezultatom \cite{chrome2025tbt}, te je Lighthouse ovdje ipak dodjelio visoku ocjenu od 100.

\textbf{Time to Interactive (TTI)} metrika bilježi učestalo visoke ocjene kod svih programskih okvira i strategija iscrtavanja (95,5\% - 96,4\%), s najboljim rezultatima na statičnoj stranici O nama (96,4\%) što je razumljivo budući da na statičnim stranicama ima najmanje izvršavanja JS koda čije bi izvršavanje produžilo ovo vrijeme.

\textbf{Speed Index (SI)} pokazuje veće varijacije ovisno o tipu stranice, s najboljim rezultatima na stranici O nama (92,5\%) i Blog post (92,3\%), dok Blog stranica zaostaje sa 89,4\%. Iz grafova kombinacija programkih okvira i strategija po metrici (slika  \ref{fig:testiranje-o-nama-postotak}, slika \ref{fig:testiranje-blog-postotak} i slika \ref{fig:testiranje-blog-post-postotak}) vidljivo je da Next.js ima prednost u odnosu na druge programske okvire i postiže bolje rezultate.

\textbf{First Contentful Paint (FCP)} - pogledom na ovu metriku (tablica \ref{tab:usporedba_metrika_po_tipu_stranice}) vidljivi su srednje dobri rezultati (76,2\% - 77,0\%). Detaljnijim pogledom na grafove (slika  \ref{fig:testiranje-o-nama-performanse-po-metrici}, slika \ref{fig:testiranje-blog-performanse-po-metrici} i slika \ref{fig:testiranje-blog-post-performanse-po-metrici}) , vidimo da Next.js, kao i kod prethodne metrike postiže bolje rezultate pri svim strategijama iscrtavanja, što pozitivno utječe na korisničko iskustvo i pokazuje bolju optimizaciju u odnosu na konkurenciju.

\textbf{Largest Contentful Paint (LCP)} poput FCP-a pokazuje niže ocjene od drugih metrika (77,4\% - 80,8\%), pri čemu Blog post stranica pokazuje najslabije ocjene (77,4\%). Za razliku od FCP-a ovdje su rezultati programskih okvira podjednako osrednji, što pokazuje da kod svih postoji prostor za poboljšanje, ali i da možda razlog leži u brzini poslužitelja Lorem Picsum servisa čija slika čini najveći vidljivi sadržaj stranice.

\textbf{Specifične kombinacije okvira i strategija} pokazuju sljedeće obrasce: Next.js s CSR strategijom dosljedno postiže najbolje FCP rezultate na svim tipovima stranica (100\% ocjena, vrijednosti od 0,79s do 0,92s). Ovo je očekivano jer CSR omogućava vrlo brzu inicijalizaciju osnovnih elemenata stranice prije učitavanja sadržaja.

Najbolje LCP rezultate postiže Nuxt.js s CSR strategijom (85\% - 88\%), ali vrijednosti od 2,59s do 2,71s još uvijek prelaze preporučeni prag od 2,5s. Ovo ukazuje na općeniti izazov optimizacije LCP metrike u modernim web aplikacijama.

Speed Index (SI) pokazuje najbolje rezultate kod Next.js kombinacija, posebno s CSR strategijom koja postiže 100\% ocjene. Vrijednosti od 1,24s do 1,75s nalaze se unutar prihvatljivih granica, što ukazuje na dobru optimizaciju progresivnog učitavanja sadržaja.

Time to Interactive (TTI) najbolje rezultate postiže kod Nuxt.js s CSR strategijom (97\% - 98\%), što sugerira da Nuxt.js ima bolje optimiziranu interaktivnost nakon inicijalizacije aplikacije.

Total Blocking Time (TBT) pokazuje izvrsne rezultate kod Next.js s CSR strategijom (100\% ocjene, vrijednosti od 8,2ms do 53,5ms), što je daleko ispod kritičnog praga od 300ms.

Cumulative Layout Shift (CLS) postiže savršene rezultate (100\% ocjene, vrijednost 0,000 - 0,020) kod većine kombinacija, što ukazuje na dobru stabilnost layouta kroz sve testirane okvire i strategije. Iznimka je Next.js s CSR strategijom na blog post stranici koji bilježi CLS vrijednost od 0,3 (40 bodova), što predstavlja jedini slučaj značajnog layoutnog pomaka u cijelom testiranju i ukazuje na potrebu za specifičnim optimizacijama ove kombinacije.

\subsection{Analiza utjecaja tipa stranice na ocjene}

Tip stranice značajno utječe na performanse različitih kombinacija okvira i strategija, što otkriva važne obrasce za praktičnu primjenu.

\textbf{Kvantitativna analiza ukupnih performansi} pokazuje jasnu hijerarhiju složenosti sadržaja. Statična stranica O nama postiže najbolje ukupne performanse s prosjekom od 91,01\% (±12,07\%), dok Blog stranica bilježi 90,33\% (±12,65\%), a stranica pojedinog blog posta najniže rezultate s 89,42\% (±14,04\%). Trend opadanja performansi (91,01\% → 90,33\% → 89,42\%) jasno ilustrira troškove povećane složenosti sadržaja.

Raspon rezultata također se povećava s složenošću stranice - od 63-100 bodova na statičnoj stranici, preko 58-100 na blog stranici, do 40-100 na stranici blog posta. Ova povećana varijabilnost (standardne devijacije od 12,07\% do 14,04\%) sugerira da složeniji sadržaj čini performanse manje predvidljivima.

\textbf{Statična stranica O nama} pokazuje najbolje ukupne performanse (87,6\% - 96,0\%) jer ne zahtijeva dinamičko dohvaćanje podataka. Next.js postiže vrhunske rezultate (96,0\%) na ovom tipu stranice, što potvrđuje optimizaciju za statični sadržaj. ISR strategija pokazuje najbolje rezultate (91,9\%) jer može u potpunosti iskoristiti statičku prirodu sadržaja.

\textbf{Stranica Blog} s listom postova predstavlja umjereno složen dinamički sadržaj. Next.js zadržava vodeću poziciju (95,1\%), ali s blago nižim ocjenama u odnosu na statičnu stranicu. SSR strategija pokazuje najbolje rezultate (91,3\%) jer omogućava efikasno server-side generiranje liste postova.

\textbf{Stranica pojedinog blog posta} predstavlja najsloženiji scenarij s najnižim ukupnim ocjenama (85,8\% - 92,3\%). Next.js i dalje vodi (92,3\%), ali s najvećim padom performansi. SSR ponovno pokazuje najbolje rezultate (90,8\%) jer omogućava dinamičko učitavanje specifičnog sadržaja posta.

Trend opadanja performansi od statičnog prema dinamičkom sadržaju (Next.js: 96,0\% → 95,1\% → 92,3\%) jasno ilustrira troškove dinamičke obrade sadržaja. Ova analiza sugerira da je izbor strategije iscrtavanja kritičan za optimizaciju performansi ovisno o prirodi sadržaja.

\subsection{Analiza rezultata dodatnih metrika}

Dodatne metrike pružaju uvid u razvojne aspekte performansi koji nisu obuhvaćeni Web Vitals metrikama.

\textbf{Vremena izgradnje} pokazuju značajne razlike među okvirima i strategijama. SSG strategija općenito zahtijeva najduža vremena izgradnje jer mora pripremiti sve stranice unaprijed, dok CSR ima najkraća vremena jer prebacuje složenost na izvršno vrijeme. Next.js demonstrira najbolju optimizaciju procesa izgradnje kroz sve strategije.

\textbf{Veličina JS paketa} varira značajno ovisno o strategiji. CSR strategija rezultira najvećim paketima jer mora uključiti svu logiku za client-side obradu, dok SSG omogućava manje pakete s optimiziranim kodom. SvelteKit pokazuje prednost u veličini paketa zbog svoje arhitekture koja eliminira nepotrebni kod.

\textbf{Konzistentnost vremena izgradnje} ključna je za razvojno iskustvo. Next.js pokazuje najveću konzistentnost, što je važno za predvidljivost razvojnog procesa. Veće varijacije kod drugih okvira mogu utjecati na produktivnost razvojnog tima.

\textbf{Korelacija između veličine paketa i vremena izgradnje} otkriva kompromise u optimizaciji. Strategije koje rezultiraju manjim paketima često zahtijevaju duža vremena izgradnje zbog dodatnih optimizacija, što predstavlja klasični trade-off između razvoja i produkcijskih performansi.

\textbf{Matrica efikasnosti} kombinira runtime performanse s razvojnim metrikama, pružajući holistički pogled na efikasnost. Next.js s ISR strategijom pokazuje najbolju ukupnu efikasnost, balansirajući performanse s praktičnošću razvoja.

\subsection{Pouzdanost izmjerenih podataka}

Analiza pouzdanosti podataka temelji se na statističkim pokazateljima varijabilnosti i konzistentnosti mjerenja.

\textbf{Standardne devijacije} kreću se od ±5,9\% do ±17,3\%, što ukazuje na umjerenu varijabilnost rezultata. Manje varijacije kod Next.js (±5,9\% - ±14,8\%) sugeriraju stabilniju i predvidljiviju platformu, dok veće varijacije kod drugih okvira mogu odražavati manje zrele optimizacije ili veću osjetljivost na testne uvjete.

\textbf{Konzistentnost preko tipova stranica} pokazuje da svi okviri održavaju relativno stabilne performanse, što ukazuje na robusnost testnih rezultata. Sistematski obrasci (npr., opadanje performansi od statičnog prema dinamičkom sadržaju) potvrđuju valjanost izmjerenih trendova.

\textbf{Reproducibilnost rezultata} kroz različite kombinacije okvira i strategija pokazuje logičke obrasce koji su u skladu s teoretskim očekivanjima. Na primjer, CSR strategija dosljedno pokazuje bolje FCP rezultate, dok SSR pokazuje prednosti kod dinamičkih stranica.

Ukupno, podaci pokazuju dovoljnu pouzdanost za izvođenje valjanih zaključaka, iako varijabilnost sugerira potrebu za kontinuiranim testiranjem u različitim uvjetima za potpunu validaciju rezultata.

\subsection{Ključna opažanja}

Analiza rezultata otkriva nekoliko ključnih opažanja koja imaju praktičnu vrijednost za razvojne timove i arhitekte web aplikacija.

\textbf{Next.js se izdvaja kao najstabilniji i najbolji okvir} kroz sve testirane scenarije, s konzistentnim performansama i najmanjom varijabilnošću rezultata. Ova prednost posebno je izražena na statičnom sadržaju, ali se održava i kroz dinamičke scenarije.

\textbf{Strategija iscrtavanja mora biti usklađena s prirodom sadržaja.} SSR strategija pokazuje najbolje rezultate za dinamički sadržaj, ISR za hibridne scenarije, a SSG za statični sadržaj. CSR strategija ima specifične prednosti za FCP i TBT metrike, ali pokazuje najveću nestabilnost s ekstremnim vrijednostima (40-100 bodova na blog post stranici).

\textbf{Kvantitativno pogoršanje performansi prema složenosti sadržaja} je sistematsko i mjerljivo - ukupne performanse opadaju za 0,68 postotnih bodova od statične na blog stranicu, a dodatnih 0,91 postotnih bodova na blog post stranicu. Ovo predstavlja ukupno smanjenje od 1,59 postotnih bodova (1,7\% relativno smanjenje) između najjednostavnije i najsloženije stranice.

\textbf{LCP metrika predstavlja najveći izazov} za sve testirane kombinacije, što ukazuje na potrebu za specifičnim optimizacijama. Ni jedna kombinacija nije dosljedno postizala izvrsne LCP rezultate, što sugerira da je ova metrika područje za buduće poboljšanje.

\textbf{Postoji jasni trade-off između runtime performansi i razvojne složenosti.} Strategije koje postižu najbolje runtime performanse (SSG, ISR) često zahtijevaju duža vremena izgradnje i složeniju konfiguraciju.

\textbf{Konzistentnost performansi jednako je važna kao i vrhunski rezultati.} Next.js pokazuje da stabilnost i predvidljivost mogu biti jednako vrijedne kao i maksimalne performanse, posebno u produkcijskim okruženjima.

\textbf{Tip stranice značajno utječe na izbor optimalne kombinacije.} Analiza pokazuje da ne postoji univerzalno najbolje rješenje - optimalni izbor ovisi o specifičnostima aplikacije i prirodi sadržaja.

Ova opažanja pružaju praktične smjernice za arhitekturalne odluke i omogućavaju razvojnim timovima da donesu informirane izbore na temelju empirijskih dokaza umjesto samo teorijskih razmatranja.
