\section{Analiza rezultata}

U ovom poglavlju slijedi detaljna analiza rezultata testiranja radnih značajki tri programska okvira (Next.js, Nuxt.js i SvelteKit) uz četiri strategije iscrtavanja (SSG, SSR, CSR i ISR) na 3 različita tipa stranica. Analiza se temelji na ocjenama  Web Vitals metrika dobivenih mjerenjem Lighthouse CLI alatom: First Contentful Paint (FCP), Largest Contentful Paint (LCP), Speed Index (SI), Time to Interactive (TTI), Total Blocking Time (TBT) i Cumulative Layout Shift (CLS) te dodatnih metrika: vremena izgradnje aplikacije, veličine JS paketa i vremena izvršavanja skripti.

\subsection{Analiza ocjena radnih značajki programskih okvira}

Evaluacija ukupnih rezultata radnih karakteristika programskih okvira otkriva jasnu hijerarhiju pri svim vrstama stranica.

\textbf{Next.js} se dosljedno pokazao kao najbolji programski okvir s ocjenama između 92,3\% i 96,0\% ovisno o tipu stranice. Okvir demonstrira najbolje rezultate na statičnom sadržaju (stranica O nama - 96,0\%), što sugerira optimizaciju za statične stranice. Blagi pad u ocjenama kod dinamičkog sadržaja (Blog - 95,1\%, Blog post - 92,3\%) sugerira određeno opterećenje povezano s obradom dinamičkih podataka, no Next.js i dalje zadržava prednost od oko 6-8\% u odnosu na konkurenciju.

\textbf{Nuxt.js} zauzima drugu poziciju s ocjenama između 88,1\% i 89,4\%. Iz rezultata vidljivo je da Nuxt.js pokazuje relativno stabilne performanse kroz različite tipove stranica, s najmanjom varijacijom između statičnog (89,4\%) i dinamičkog sadržaja (88,1\% - 88,5\%). Ova konzistentnost ukazuje na dobro uravnoteženu arhitekturu koja se jednako dobro nosi s različitim tipovima sadržaja.

\textbf{SvelteKit} na testovima bilježi neznatno niži rezultat od Nuxt.js-a (87,4\% - 87,8\%). Baš kao i Nuxt.js, SvelteKit održava konstantne performanse neovisno o vrsti stranice. Manje ocjene mogu se objasniti time da je SvelteKit mlađi programski okvir koji je još uvijek u razvoju i procesu optimizacije.

Kad su u pitanju standardne devijacije, Next.js demonstrira najbolje ocjene dosljednosti (±7,0\% do ±14,8\%), dok su preostali okviri podložniji većim varijacijama (±13,2\% do ±14,1\%). To ukazuje na zrelost Next.js-a kao stabilnijeg i pouzdajnijeg programskog okvira.

\subsection{Analiza ocjena strategija iscrtavanja}

Analiza strategija iscrtavanja otkriva zanimljive obrasce koji se razlikuju ovisno o tipu stranice i prirodi sadržaja.

\textbf{Incremental Static Regeneration (ISR)} ostvaruje najbolje rezultate na statičnoj stranici O nama (91,9\%), što i nije iznenađujuće s obzirom na prednosti statičke regeneracije uz mogućnost povremenog osvježavanja sadržaja. Na dinamičnim stranicama ISR se drži vrlo stabilno, s rezultatima između 90,6\% i 90,7\%, što potvrđuje njegovu fleksibilnost.

\textbf{Server-Side Rendering (SSR)} pokazao je odlične rezultate, osobito kod prikaza dinamičkog sadržaja na stranicama Blog (91,3\%) i Blog post (90,8\%). Zanimljivo, upravo na statičnoj stranici O nama SSR je ostvario gotovo najbolji rezultat (91,7\%), što govori da mu tip sadržaja ne predstavlja značajnu prepreku.

\textbf{Static Site Generation (SSG)} zadržava dobre ocjene na svim testiranim stranicama, pri čemu se najbolje pokazao na Blogu i O nama (oko 90,6\%–90,7\%). Ta ujednačenost ukazuje na glavnu prednost SSG-a: sadržaj je u potpunosti pripremljen tijekom izgradnje stranice, što rezultira predvidljivim performansama, bez obzira na složenost same stranice.

\textbf{Client-Side Rendering (CSR)} s druge strane, pokazuje znatno veće varijacije. Na O nama stranici postiže solidnih 89,7\%, dok na složenijim dinamičkim stranicama, poput pojedinačnih blog postova, padne i na 85,8\%. Štoviše, zabilježena je i najniža pojedinačna ocjena u cijelom testiranju — samo 40 bodova na blog post stranici (Next.js). Iz grafova
na slici \ref{fig:testiranje-blog-post-performanse-po-metrici} i slici \ref{fig:testiranje-blog-post-postotak} vidljivo je da Next.js programski okvir postiže vrlo loš rezultat metrike CLS specifično na blog post stranici. Ova loša ocjena utječe na cjelokupni rezultat učinkovitosti ove strategije. S obzirom da se na stranici Blog koja je također dinamičnog sadržaja ne pojavljuje ova anomalija, da se zaključiti da arhitektura koda ove stranice nije optimalno strukturirana za ovu kombinaciju programskog okvira i strategije iscrtavanja.

Kad se promatraju standardne devijacije, SSR i ISR imaju najmanje varijacije (oko ±11,5\% do ±12,9\%), dok CSR pokazuje najveće (±13,0\% do ±17,3\%). To potvrđuje da su server-side strategije u prosjeku predvidljivije i stabilnije.

\subsection{Analiza rezultata pojedinih metrika - Web Vitals}

Detaljnom analizom pojedinačnih Web Vitals metrika možemo identificirati specifične snage i slabosti različitih kombinacija okvira i strategija.

\textbf{Cumulative Layout Shift (CLS)} postiže visoke ukupne rezultate s ocjenama od 95,0\% do 100,0\%, a blog stranica postiže savršenu ocjenu (100,0\%). Drastično veliki pad ocjene ove metrike bilježi Next.js s CSR strategijom na blog post stranici, gdje je zabilježena vrijednost od 0,3 (40 bodova). Ovaj pad ukazuje na značajan problem s pomakom rasporeda elemenata na stranici, što može negativno utjecati na korisničko iskustvo.

\textbf{Total Blocking Time (TBT)} - odlični rezultati (99,9\% - 100,0\%) pri svim kobinacijama demonstriraju dobru optimizaciju programskih okvira. Pogledom na grafove sa vrijednostima ove metrike  (slika \ref{fig:testiranje-o-nama-vrijednosti}, slika \ref{fig:testiranje-blog-vrijednosti} i slika \ref{fig:testiranje-blog-post-vrijednosti}) uočavamo da Next.js programski okvir ima povišeno vrijeme blokiranja u odnosu na druge programske okvire, no pogled na grafove sa ukupnom ocjenom (slika \ref{fig:testiranje-o-nama-postotak}, slika \ref{fig:testiranje-blog-postotak} i slika \ref{fig:testiranje-blog-post-postotak}) otkriva da se ova odstupanja i dalje smatraju odličnim rezultatom \cite{chrome2025tbt}, te je Lighthouse ovdje ipak dodjelio visoku ocjenu od 100.

\textbf{Time to Interactive (TTI)} metrika bilježi učestalo visoke ocjene kod svih programskih okvira i strategija iscrtavanja (95,5\% - 96,4\%), s najboljim rezultatima na statičnoj stranici O nama (96,4\%) što je razumljivo budući da na statičnim stranicama ima najmanje izvršavanja JS koda čije bi izvršavanje produžilo ovo vrijeme.

\textbf{Speed Index (SI)} pokazuje veće varijacije ovisno o tipu stranice, s najboljim rezultatima na stranici O nama (92,5\%) i Blog post (92,3\%), dok Blog stranica zaostaje sa 89,4\%. Iz grafova kombinacija programkih okvira i strategija po metrici (slika  \ref{fig:testiranje-o-nama-postotak}, slika \ref{fig:testiranje-blog-postotak} i slika \ref{fig:testiranje-blog-post-postotak}) vidljivo je da Next.js ima prednost u odnosu na druge programske okvire i postiže bolje rezultate.

\textbf{First Contentful Paint (FCP)} - pogledom na ovu metriku (tablica \ref{tab:usporedba_metrika_po_tipu_stranice}) vidljivi su srednje dobri rezultati (76,2\% - 77,0\%). Detaljnijim pogledom na grafove (slika  \ref{fig:testiranje-o-nama-performanse-po-metrici}, slika \ref{fig:testiranje-blog-performanse-po-metrici} i slika \ref{fig:testiranje-blog-post-performanse-po-metrici}) , vidimo da Next.js, kao i kod prethodne metrike postiže bolje rezultate pri svim strategijama iscrtavanja, što pozitivno utječe na korisničko iskustvo i pokazuje bolju optimizaciju u odnosu na konkurenciju.

\textbf{Largest Contentful Paint (LCP)} poput FCP-a pokazuje niže ocjene od drugih metrika (77,4\% - 80,8\%), pri čemu Blog post stranica pokazuje najslabije ocjene (77,4\%). Za razliku od FCP-a ovdje su rezultati programskih okvira podjednako osrednji, što pokazuje da kod svih postoji prostor za poboljšanje, ali i da možda razlog leži u brzini poslužitelja Lorem Picsum servisa čija slika čini najveći vidljivi sadržaj stranice.

\subsection{Analiza utjecaja tipa stranice na ocjene}

Tip stranice značajno utječe na performanse različitih kombinacija okvira i strategija, što otkriva važne obrasce za praktičnu primjenu.

Statična stranica O nama postiže najbolje ukupne performanse s prosjekom od 91,01\% (±12,07\%), dok Blog stranica bilježi 90,33\% (±12,65\%), a stranica pojedinog blog posta najniže rezultate s 89,42\% (±14,04\%). Trend opadanja performansi (91,01\% → 90,33\% → 89,42\%) jasno ilustrira troškove povećane složenosti sadržaja od statičnog prema dinamičnom.

Raspon rezultata također se povećava s složenošću stranice - od 63-100 bodova na statičnoj stranici, preko 58-100 na blog stranici, do 40-100 na stranici blog posta. Ova povećana varijabilnost (standardne devijacije od 12,07\% do 14,04\%) sugerira da složeniji sadržaj čini performanse manje predvidljivima.

\textbf{Statična stranica O nama} pokazuje najbolje ukupne performanse (87,6\% - 96,0\%) jer ne zahtijeva dinamičko dohvaćanje podataka. Next.js postiže vrhunske rezultate (96,0\%) na ovom tipu stranice, što potvrđuje optimizaciju za statični sadržaj. ISR strategija pokazuje najbolje rezultate (91,9\%) jer može u potpunosti iskoristiti statičku prirodu sadržaja.

\textbf{Stranica Blog} s listom postova predstavlja umjereno složen dinamički sadržaj. Next.js zadržava vodeću poziciju (95,1\%), ali s blago nižim ocjenama u odnosu na statičnu stranicu. SSR strategija pokazuje najbolje rezultate (91,3\%) jer omogućava efikasno server-side generiranje liste postova.

\textbf{Stranica pojedinog blog posta} predstavlja najsloženiji scenarij s najnižim ukupnim ocjenama (85,8\% - 92,3\%). Next.js i dalje vodi (92,3\%), ali s najvećim padom performansi. SSR ponovno pokazuje najbolje rezultate (90,8\%) jer omogućava dinamičko učitavanje specifičnog sadržaja posta.

\bigskip
Ova analiza sugerira da je izbor strategije iscrtavanja kritičan za optimizaciju performansi ovisno o prirodi sadržaja, no vrlo je i važan odabir programskog okvira kada su u pitanju sirove performanse.

\subsection{Analiza rezultata dodatnih metrika}

Dodatne metrike pružaju uvid u razvojne aspekte performansi koji nisu obuhvaćeni Web Vitals metrikama. Analizom vremena izgradnje, veličine JS paketa i vremena izvršavanja skripti dobivamo potpuniju sliku o efikasnosti pojedinih okvira i strategija.

\textbf{Vremena izgradnje} pokazuju značajne razlike među okvirima (slika \ref{fig:build_times_heat_map} i slika  \ref{fig:overall_framework_build_performance}). SvelteKit se ističe kao najbrži u svim strategijama, s prosječnim vremenima od 8,7 do 14,3 sekunde. Nuxt.js je drugi s konzistentnim vremenima oko 20-22 sekunde. Next.js pokazuje najveće varijacije; dok su SSR, SSG i ISR strategije usporedive s Nuxt.js, CSR strategija drastično odskače s vremenom od 42,3 sekunde, što ga čini najsporijim u toj kategoriji. Suprotno očekivanjima, CSR strategija nije uvijek najbrža za izgradnju, što je vidljivo kod Next.js i Nuxt.js gdje je ona najsporija. Vrijednosti se temelje na prosjeku 3 mjerenja izgradnje prilikom postavljanja aplikacija na Vercel platformu.

\textbf{Veličina JS paketa} ključna je za performanse učitavanja stranice. SvelteKit se izdvaja s najmanjim paketima (oko 58-59 kB), što potvrđuje njegovu reputaciju okvira koji proizvodi visoko optimiziran kod. Nuxt.js slijedi s paketima od 118 kB za sve strategije. Next.js ima najveće pakete, koji variraju od 135 kB do 154 kB, pri čemu CSR strategija generira najveći paket (slika \ref{fig:bundle_size_distribution_by_framework} i slika  \ref{fig:average_bundle_size_by_strategy}). Ovi podaci potvrđuju da SvelteKit ima značajnu prednost u pogledu veličine paketa. Podaci o veličini paketa dobiveni su i potvrđeni uvidom u ukupno preuzeti JS kod prilikom učitavanja stranice kroz Chrome DevTools.

\textbf{Vrijeme izvršavanja skripti (Scripting time)} direktno utječe na interaktivnost stranice. SvelteKit konzistentno pokazuje najkraća vremena izvršavanja, što ga čini najefikasnijim u pogledu klijentske obrade (slika \ref{fig:average_scripting_performance_times}). Next.js i Nuxt.js imaju značajno duža vremena izvršavanja, posebno pri korištenju CSR strategije, gdje Nuxt.js bilježi najduža vremena. Činjenica da CSR strategija ima najduže vrijeme izvršavanja skripti je očekivana budući da programski okvir mora ispuniti cijeli html dokument koji je u trenutku učitavanja stranice prazan, dohvatiti vanjske podatke i iscrtati dokument. Jednostavno rečeno kod CSR strategije JS kod ima više posla kojeg treba obaviti u odnosu na druge strategije gdje se jedan dio tog posla događa na poslužitelju. Što se tiče samih programskih okvira, SvelteKit je brži od konkurencije jer ne koristi virtualni DOM, već generira optimizirani JavaScript kod koji direktno manipulira DOM-om, čime se smanjuje složenost i povećava efikasnost izvršavanja skripti \cite{svelte2019reactivity}.


\textbf{Odnos između veličine paketa i vremena izgradnje} pokazuje snažnu pozitivnu korelaciju (r=0.868), što ukazuje na to da veći paketi općenito zahtijevaju duža vremena izgradnje (slika \ref{fig:bundle_size_vs_build_time_correlation}). Analiza otkriva jasnu hijerarhiju programskih okvira: SvelteKit ostvaruje najmanju veličinu paketa (58-58.9 kB) s najkraćim vremenom izgradnje (8.7-14 sekundi), Nuxt.js zauzima srednju poziciju (118 kB, 20-22 sekunde), dok Next.js, usprkos najvećim paketima (135-154 kB) i najduljim vremenima izgradnje (22.7-42.3 sekunde), ostvaruje najviše performanse.
Zaključak koji se nameće je da rastom veličine JS paketa raste i vrijeme izgradenje, no i da je za postizanje većih performansi potrebno prihvatiti ovaj kompromis. Next.js ide u tom smjeru, dok SvelteKit i Nuxt.js nude brža vremena izgradnje s manjim paketima, ali s nešto nižim performansama.

\textbf{Odnos između vremena skriptiranja i veličine JS paketa} prikazan je na grafu na slici \ref{fig:performance_vs_bundle_size_tradeoff}. Ovaj graf pokazuje da veći JS paketi ne rezultiraju nužno dužim vremenima izvršavanja skripti, što Next.js lijepo demonstrira te sugerira da optimizacija koda može kompenzirati povećanje veličine paketa. SvelteKit uvjerljivo nadmašuje konkurenciju kako u veličini JS paketa, tako i u vrlo niskom vremenu skriptiranja. Nuxt ovdje malo zaostaje, no i dalje je unutar vrlo prihvatljivog vremena skriptiranja.
\subsection{Pouzdanost izmjerenih podataka}

Analiza pouzdanosti podataka temelji se na statističkim pokazateljima varijabilnosti i konzistentnosti mjerenja te je u skladu s nalazima iz izvještaja "Lighthouse Metric Variability and Accuracy" (javno dostupan Google Doc \cite{lh_variability2025}) koji identificira mrežu, hardver, poslužitelj i nondeterminističko ponašanje preglednika kao glavne izvore varijabilnosti.

\textbf{Opažene varijacije -} U našim skupovima podataka standardne devijacije kreću se od ±5,9\% do ±17,3\%, što ukazuje na umjerenu do povišenu varijabilnost ovisno o kombinaciji okvira/strategije. Manje varijacije kod Next.js (±5,9\% - ±14,8\%) potvrđuju njegovu veću stabilnost, dok veće varijacije kod CSR kombinacija i nekih Nuxt/SvelteKit scenarija odražavaju osjetljivost na implementacijske razlike i testne uvjete (kao što je mrežno kašnjenje ili različito ponašanje klijentskog JS koda).

Nalazi u spomenutom dokumentu potvrđuju da pojedine metrike (npr. LCP, TTI) mogu imati znatnu varijabilnost između mjerenja te da za robustnu procjenu treba koristiti više ponavljanja i kontrolirano okruženje.

\textbf{Preporučeni protokol mjerenja -}
Googleov dokument \cite{lh_variability2025} preporučuje sljedeće korake za pouzdano mjerenje performansi:
- Izvršiti najmanje 3 mjerenja po konfiguraciji; za metrike ili konfiguracije s visokom varijabilnošću (LCP, anomalije u CSR) preporučljivo je 5 ili više mjerenja.
- Pri izvještavanju prikazati: broj mjerenja (n), srednju vrijednost (mean), medijan, standardnu devijaciju, te minimalnu i maksimalnu vrijednost.
- Koristiti dosljedno, kontrolirano okruženje: isto hardversko okruženje, isključen/umjeren promet u pozadini, stabilna mrežna veza (po mogućnosti žičana ili lokalno simulirana), i fiksno throttling podešenje pri korištenju Lighthouse/DevTools.

Testiranja na kojima se temelji ovaj rad  provedena su u skladu s navedenim Googleovim smjernicama. Za svaki testni scenarij urađeno je 10 mjerenja, na vrlo stabilnoj internetskoj vezi, uz fiksno throttling podešenje. Uz standardnu devijaciju, u izvještajima su zabilježene i minimalne i maksimalne vrijednosti podataka \footnote{Svi izvještaji i izmjereni podaci kao i testna Node.js skripta lighthouse-reporter dostupni su na repozitoriju : \url{https://github.com/AlphaActual/lighthouse-reporter}}.


\subsection{Ključna opažanja}

Analiza rezultata otkriva nekoliko ključnih opažanja koja imaju praktičnu vrijednost za razvojne timove i arhitekte web aplikacija.

\textbf{Next.js se izdvaja kao najstabilniji i najbolji okvir} kroz sve testirane scenarije, s konzistentnim performansama i najmanjom varijabilnošću rezultata. Ova prednost posebno je izražena na statičnom sadržaju, ali se održava i kroz dinamičke scenarije.

\textbf{Strategija iscrtavanja mora biti usklađena s prirodom sadržaja.} SSR strategija pokazuje najbolje rezultate za dinamički sadržaj, ISR za hibridne scenarije, a SSG za statični sadržaj. CSR strategija ima specifične prednosti za FCP i TBT metrike, ali pokazuje najveću nestabilnost i varijabilnost.

\textbf{Kvantitativno pogoršanje performansi prema složenosti sadržaja} je sistematsko i mjerljivo - ukupne performanse opadaju za 0,68 postotnih bodova od statične na blog stranicu, a dodatnih 0,91 postotnih bodova na blog post stranicu. Ovo predstavlja ukupno smanjenje od 1,59 postotnih bodova (1,7\% relativno smanjenje) između najjednostavnije i najsloženije stranice.

\textbf{LCP metrika predstavlja najveći izazov} za sve testirane kombinacije, što ukazuje na potrebu za specifičnim optimizacijama. Ni jedna kombinacija nije dosljedno postizala izvrsne LCP rezultate, što sugerira da je ova metrika područje za buduće poboljšanje.

\textbf{Postoji jasni kompromis između performansi i razvojne složenosti.} Strategije koje postižu najbolje ocjene (SSG, ISR) često zahtijevaju duža vremena izgradnje i složeniju konfiguraciju.

\textbf{Konzistentnost performansi jednako je važna kao i vrhunski rezultati.} Next.js pokazuje da stabilnost i predvidljivost mogu biti jednako vrijedne kao i maksimalne performanse, posebno u produkcijskim okruženjima.

\textbf{Tip stranice značajno utječe na izbor optimalne kombinacije.} Analiza pokazuje da ne postoji univerzalno najbolje rješenje - optimalni izbor ovisi o specifičnostima aplikacije i prirodi sadržaja.

\textbf{Dodatne metrike otkrivaju važne razvojne kompromise.} SvelteKit se ističe dramatičnom prednošću u veličini JS paketa (58kB vs 118-154kB konkurenata) i najkraćim vremenima izgradnje, što ga čini idealnim za projekte gdje su brzina razvoja i mala veličina paketa prioritet. Snažna korelacija između veličine paketa i vremena izgradnje (r=0.868) potvrđuje da performanse dolaze uz cijenu razvojne složenosti.

\textbf{Identificirani su specifični ograničavajući faktori.} Next.js CSR kombinacija na blog post stranici pokazuje anomalno ponašanje (CLS = 0.3), što ukazuje na važnost temeljitog testiranja specifičnih kombinacija i prilagodbu arhitekture koda određenoj strategiji iscrtavanja. Vanjski faktori poput brzine Lorem Picsum servisa mogu ograničiti LCP performanse neovisno o arhitekturnim odlukama.

\textbf{Razlika između teorijskih i praktičnih performansi je značajna.} Dok Next.js teorijski ima najveće JS pakete i najdulja vremena izgradnje, u praksi postiže najbolje ukupne performanse, što naglašava važnost empirijskog testiranja nad spekulativnim optimizacijama.

Ova opažanja pružaju praktične smjernice za arhitekturalne odluke i omogućavaju razvojnim timovima da donesu informirane izbore na temelju empirijskih dokaza umjesto samo teorijskih razmatranja.
