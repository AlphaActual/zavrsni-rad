\selectlanguage{croatian}
\begin{abstract}

Odabir optimalne strategije iscrtavanja web aplikacije ključan je za postizanje dobrih performansi i korisničkog iskustva. Ovaj rad usporedno analizira četiri strategije iscrtavanja (CSR, SSR, SSG, ISR) u tri popularna programska okvira za izradu web aplikacija (Next.js, Nuxt.js i SvelteKit). U svakom od programskih okvira implementirana je identična demo aplikacija sa stranicama statičnog i dinamičnog sadržaja, te su na njima vršeni testovi ključnih metrika performansi prikupljeni Lighthouse CLI alatom u kontroliranim mrežnim uvjetima. Analiza rezultata otkriva prednosti i mane svake strategije iscrtavanja ovisno o tipu stranice, te koji programski okviri postižu najbolje performanse u kojim strategijama iscrtavanja.

\end{abstract}
\begin{small}
\textbf{Ključne riječi} : CSR, SSR, SSG, ISR, Next.js, Nuxt.js, SvelteKit, performanse
\end{small}

\bigskip

\selectlanguage{english}
\begin{abstract}

Selecting the optimal rendering strategy for a web application is crucial for achieving good performance and user experience. This paper provides a comparative analysis of four rendering strategies (CSR, SSR, SSG, ISR) across three popular web application frameworks (Next.js, Nuxt.js, and SvelteKit). An identical demo application containing both static and dynamic content pages was implemented in each framework, and key performance metrics were tested using the Lighthouse CLI tool under controlled network conditions. The results highlight the advantages and disadvantages of each rendering strategy depending on the page type, as well as which frameworks achieve the best performance under specific rendering strategies.

\end{abstract}
\begin{small}
\textbf{Keywords} : CSR, SSR, SSG, ISR, Next.js, Nuxt.js, SvelteKit, performance
\end{small}
