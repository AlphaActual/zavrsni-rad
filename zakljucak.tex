\section{Zaključak}

Ovaj rad predstavlja sveobuhvatnu empirijsku analizu performansi tri vodeća programska okvira za razvoj web aplikacija (Next.js, Nuxt.js i SvelteKit) kroz četiri različite strategije iscrtavanja (CSR, SSR, SSG i ISR). Kroz sustavno testiranje 360 kombinacija i analizu 12 ključnih metrika performansi, istraživanje pruža konkretne dokaze koji mogu usmjeriti arhitekturalne odluke u modernom web razvoju.

\subsection{Glavni nalazi istraživanja}

\textbf{U hijerarhiji programskih okvira} Next.js nedvosmisleno je pozicioniran kao vodeći programski okvir s ukupnim ocjenama između 92,3\% i 96,0\% ovisno o tipu stranice. Ova prednost nije samo kvantitativna već i kvalitativna - Next.js demonstrira najveću stabilnost s najmanjim standardnim devijacijama (±7,0\% do ±14,8\%) što ga uz odličnu dokumentaciju i široku podršku čini najboljim i najpouzdanijim izborom za produkcijske aplikacije. Nuxt.js zauzima drugu poziciju s konzistentnim ocjenama između 88.1 i 89.4\%, dok SvelteKit ne zaostaje mnogo sa vrlo respektabilnim rezultatima od 87,4\% do 87,8\% (slika \ref{fig:ukupne_ocjene_radnih_znacajki}).

\textbf{Optimalne strategije iscrtavanja prema vrsti sadržaja -} Analiza otkriva jasne obrasce u pogledu optimalnih strategija za različite tipove sadržaja. Inkrementalna statička regeneracija (ISR) pokazuje se kao najbolja strategija za statični sadržaj (91,9\% na stranici O nama), kombinirajući prednosti statičke generacije s fleksibilnošću dinamičkog ažuriranja. Za dinamički sadržaj, SSR strategija ostvaruje najbolje rezultate (91,3\% na blog stranicama), dok SSG zadržava solidne performanse kroz sve tipove stranica zahvaljujući predvidljivosti unaprijed pripremljenog sadržaja. CSR strategija, unatoč svojim teoretskim prednostima, pokazuje najveću varijabilnost i najniže ukupne performanse, osobito na složenijim stranicama.

\textbf{Sistematsko pogoršanje performansi prema složenosti -} Istraživanje kvantificira troškove povećane složenosti sadržaja: ukupne performanse opadaju s 91,01\% na statičnoj stranici preko 90,33\% na blog stranici do 89,42\% na stranicu pojedinog blog posta. Ovo predstavlja ukupno smanjenje od 1,59 postotnih bodova ili 1,7\% relativno smanjenje, što je statistički značajno i praktički relevantno za korisničko iskustvo.

\textbf{Specifične slabosti postojećih rješenja -} Largest Contentful Paint (LCP) metrika identificirana je kao najveći izazov za sve testirane kombinacije, s ocjenama od 77,4\% do 80,8\%. Ova konzistentna slabost ukazuje na sistemske probleme koji nadilaze izbor okvira ili strategije, možda vezane uz optimizaciju slika ili vanjskih resursa. Dodatno, identificirana je specifična anomalija u Next.js CSR kombinaciji na blog post stranici gdje je Cumulative Layout Shift dosegao problematičnu vrijednost od 0,3, što rezultira ocjenom od samo 40 bodova. Ova anomalija naglašava važnost temeljitog testiranja specifičnih kombinacija i prilagodbu arhitekture koda određenoj strategiji iscrtavanja.

\subsection{Praktične implikacije za razvoj}

Za projekte gdje su performanse kritičan faktor, Next.js se nameće kao jasni izbor zbog svoje stabilnosti i vrhunskih rezultata. Nuxt.js predstavlja solidnu alternativu za Vue.js ekosustav s predvidljivim performansama. SvelteKit, unatoč mladosti, pokazuje značajan potencijal, osobito za projekte gdje su veličina JS paketa (58kB vs 118-154kB konkurenata) i brzina razvoja prioritet.

\textbf{Optimizacija prema vrsti sadržaja -} Statični sadržaj poput marketing stranica i dokumentacije trebao bi koristiti ISR ili SSG strategije za maksimalne performanse. Dinamički sadržaj kao što su forumi, dashboards ili real-time aplikacije trebaju SSR pristup. CSR strategiju treba rezervirati za specifične slučajeve gdje je visoka interaktivnost važnija od početnih performansi učitavanja.

\textbf{Razvojni kompromisi.} Istraživanje otkriva snažnu korelaciju između veličine JS paketa i vremena izgradnje (r=0.868), te nadalje sugerira da performanse dolaze uz cijenu razvojne složenosti. SvelteKit nudi najbrža vremena izgradnje (8,7-14,3 sekunde) i najmanje JS pakete, što ga čini idealnim za iterativni razvoj, dok Next.js zahtijeva duža vremena izgradnje (do 42,3 sekunde za CSR) i pruža superiorne performanse unatoč najvećim veličinama JS paketa na testovima.

\subsection{Ograničenja istraživanja}

Rezultati ovog istraživanja trebaju se interpretirati u kontekstu određenih ograničenja. Testiranje je provedeno na demo aplikaciji relativno jednostavne arhitekture, što možda ne odražava u potpunosti složenost realnih produkcijskih aplikacija. Korištenje vanjskih servisa poput Lorem Picsuma i JSONPlaceholdera moglo je utjecati na određene metrike, osobito LCP. Testiranje je također ograničeno na Vercel platformu, što je bilo nužno za podršku svih strategija iscrtavanja, ali može utjecati na generalizaciju rezultata za druge hosting platforme.

\subsection{Smjernice za buduća istraživanja}

Ovo istraživanje otvara nekoliko pravaca za buduće studije. Potrebna su testiranja na složenijim aplikacijama s različitim arhitekturnim obrascima kako bi se validirali nalazi u realnim scenarijima. Analiza performansi na različitim hosting platformama bila bi vrijedna za razumijevanje utjecaja infrastrukture na rezultate. Dodatno, longitudinalna studija koja prati evoluciju performansi kroz verzije okvira mogla bi pružiti uvid u trendove optimizacije.

Posebno područje interesa predstavlja dublja analiza LCP optimizacije budući da je identificirana kao univerzalni problem. Istraživanje različitih pristupa optimizaciji slika, lazy loading strategija i CDN konfiguracija moglo bi pružiti praktične smjernice za poboljšanje ove kritične metrike.

\subsection{Završne napomene}

Rezultati ovog istraživanja potvrđuju da ne postoji univerzalno najbolje rješenje u modernom web razvoju. Optimalne kombinacije programskih okvira i strategija iscrtavanja ovise o specifičnostima aplikacije, prirodi sadržaja i prioritetima razvoja. Međutim, empirijski podaci jasno pokazuju hijerarhiju opcija i pružaju konkretne smjernice za donošenje informiranih odluka.

Next.js se izdvaja kao najzreliji i najstabilniji izbor za kritične aplikacije, dok SvelteKit nudi inovativni pristup s obećavajućim rezultatima za buduće projekte. Strategije iscrtavanja moraju biti pažljivo odabrane prema prirodi sadržaja, pri čemu SSR dominira kod dinamičkog sadržaja, a ISR/SSG kod statičnog.

Važnost empirijskog testiranja ne može se dovoljno naglasiti. Dok teorijske prednosti određenih pristupa mogu zvučati uvjerljivo, stvarne performanse ponekad odstupaju od očekivanja. Ovaj rad demonstrira vrijednost sustavnog mjerenja i analize u procesu donošenja arhitekturalnih odluka.

Konačno, dinamična priroda web tehnologija zahtijeva kontinuirano preispitivanje i ažuriranje spoznaja. Kako programski okviri evoluiraju i nove strategije iscrtavanja nastaju, buduća istraživanja trebaju nastaviti pružati empirijske temelje za evoluciju web razvoja i najboljih praksi.
